\documentclass[german,a4paper]{llncs}

\usepackage{german}
\usepackage{palatino}
\usepackage[T1]{fontenc}
\usepackage[utf8x]{inputenc}
\usepackage{graphicx, wrapfig} 							% Grafikpaket und Grafik im Textfluß
\usepackage{bibgerm}
\usepackage[footnote, nolist]{acronym}                   % Abkürzungen
\usepackage{tabularx,longtable, multirow} 				% Tabellen
\renewcommand\arraystretch{1.4}                          % Tabellen Zeilenhöhe
\newcolumntype{C}[1]{>{\hsize=#1\hsize\centering\arraybackslash}X} % Spalte zentrieren
\newcolumntype{L}[1]{>{\hsize=#1\hsize\raggedright\arraybackslash}X}%
% Seitenränder
\usepackage{geometry}
\geometry{verbose,a4paper,tmargin=5.2cm,bmargin=5.2cm,lmargin=4.4cm,rmargin=4.4cm}

% Kopf- und Fusszeile
\usepackage{fancyhdr}
\pagestyle{fancy}
\renewcommand{\headrulewidth}{0pt}
\renewcommand{\footrulewidth}{0pt}
\fancyhead[EL]{\textsc{Fabian Schneider}}				% Hier bitte den/die Autor(en) angeben
\fancyhead[OR]{\textsc{CDR}}						% Hier bitte den Dokumententitel angeben
\fancyfoot[EC,OC]{}

\usepackage{listingsutf8, color}


\lstset{
   backgroundcolor=\color[rgb]{0.95, 0.95, 0.95},
   tabsize=2,
%   rulecolor=,
   basicstyle=\ttfamily,
   extendedchars=true,
   showstringspaces=false,
   showspaces=false,
   numbers=left,
   numberstyle=\footnotesize,
   numbersep=5pt,
   breaklines=true,
   showtabs=false,
   captionpos=b,
   keywordstyle=\color[rgb]{1.0,0,0},
   commentstyle=\color[rgb]{0.133,0.545,0.133},
   stringstyle=\color[rgb]{0.1,0.082,0.88},
%   basicstyle=\footnotesize,
%   extendedchars=\true,
   inputencoding=utf8
}

\definecolor{lightgray}{rgb}{.9,.9,.9}
\definecolor{darkgray}{rgb}{.4,.4,.4}
\definecolor{purple}{rgb}{0.65, 0.12, 0.82}
\lstdefinelanguage{JavaScript}{
  keywords={typeof, new, true, false, catch, function, return, null, catch, switch, var, if, in, while, do, else, case, break},
  keywordstyle=\color{blue}\bfseries,
  ndkeywords={class, export, boolean, throw, implements, import, this},
  ndkeywordstyle=\color{darkgray}\bfseries,
  identifierstyle=\color{black},
  sensitive=false,
  comment=[l]{//},
  morecomment=[s]{/*}{*/},
  commentstyle=\color{purple}\ttfamily,
  stringstyle=\color{red}\ttfamily,
  morestring=[b]',
  morestring=[b]"
}
\renewcommand*\lstlistingname{Beispielcode}
\usepackage{hyperref}									% Links ins Internetz
\begin{document}

% Titel
\title{CDR}										% Hier bitte den Dokumententitel angeben
\subtitle{Contexts and Dependency Remoting}							% Hier bitte den Untertitel angeben
\author{Fabian Schneider}					    			% Hier bitte den/die Autor(en) angeben; mehrere bitte durch Komma trennen
\institute{Universität Ulm, Institut für Datenbanken und Informationssysteme\\
           \email{Fabian-2.Schneider@uni-ulm.de}}		% Hier bitte die EMail-Adresse(n) des/der Autors/Autoren angeben
           
\maketitle

\begin{acronym}
% \acro{ISO}{International Organization for Standardization}
\end{acronym}
% Zusammenfassung
\begin{abstract}
 
\end{abstract}

% ab hier beginnt der eigentlich Text

\section{Einleitung\label{sec:introduction}}
In verteilten Softwarearchitekturen ist es wichtig, dass die einzelnen Komponenten flexibel auf verschiedenen Server bereitgestellt werden können. 
Je nach Anwendungsfall müssen beispielsweise alle Komponenten auf einem Server oder eine einzelne Komponente auf einem speziellen Server bereitgestellt werden. 
Die Nutzung einer Komponenten innerhalb der Geschäftslogik der Anwendung soll dabei frei von den Details der Verteilung sein.\\
In einer Objektorientierten Sprache wie Java sind folgende drei Ziele zu erreichen:
\begin{description}
\item[Methodensignaturen] Die Parameter und der Rückgabewert der Methoden sollen keine Details über die Art der Bereitstellung aufweisen.
\item[Serialisierung] Für eine flexibel Verteilung der Komponenten müssen die Objekte die als Parameter und Rückgabewert genutzt werden, serialisierbar sein. Die Details der Serialisierung sollen nicht Teil der Geschäftslogik sein. 
\item[Exceptions] Auch die Behandlung von Ausnahmefehlern muss frei von Bereitstellungsdetails bleiben. Zudem muss sichergestellt werden, dass auch bei verteilten Bereitstellungsvarianten die speziellen Fehlerklassen der Anwendung erhalten bleiben.    
\end{description}
Das im Folgenden vorgestellte Konzept \ac{CDR}, versucht diese Anforderungen für die \ac{Java EE}"=Plattform umzusetzen.

\input{sCDI}

\input{sREST}

\input{sCDR}

\section{Fazit und Ausblick}
\begin{table}[t]
\setlength{\tabcolsep}{10pt}
\centering
\caption{\label{tab:zielerreichung}Zielerreichung durch die Beispielimplementierung.}
\begin{tabular}{lc}
Methodensignaturen & \checkmark \\
Serialisierung     & \checkmark \\
Exceptions         & \checkmark / 0\\
\end{tabular}
\vspace{-10pt}
\end{table}
Das grundsätzliche Ziel des \ac{CDR}"=Entwurfsmusters ist es, die Geschäftslogik von den Details der Bereitstellung zu trennen.
Tabelle \ref{tab:zielerreichung} gibt einen Überblick darüber, für welche der in Kapitel \ref{sec:introduction} definierten Bereiche dieses Ziel erreicht werden konnten.
Die Bereiche \textbf{Methodensignaturen} und \textbf{Serialisierung} sind in der Beispielimplementierung frei von Bereitstellungsdetails. 
Das gesetzte Ziel konnte hier erreicht werden.\\
Einschränkungen gibt es bei der Fehlerbehandlung. Fehlertypen können zwar auch bei entfernter Bereitstellung erhalten bleiben, aber Kapitel \ref{sec:implementation} zeigt, dass für eine erfolgreiche Abbildung der Fehlertypen auf \ac{HTTP}"=Statuscodes und wieder zurück, die Fehlertypen den Typ \textit{RuntimeException} spezialisieren müssen.
Da \textit{RuntimeExceptions} in Java eine besondere Semantik haben, schränkt dies die Fehlerbehandlung der Anwendung ein und ist zu vermeiden. 
Den Ursprung hat diese Einschränkung in der genutzten \textit{resteasy}"=Version. Andere \ac{JAX-RS}"=Implementierungen haben diese Einschränkung nicht. Siehe dazu auch \cite{cxf}.\\
Die Beispielimplementierung legt zum Zeitpunkt der Bereitstellung der Anwendung fest, ob eine von CDR aktivierte Komponente lokal oder entfernt referenziert wird. 
Mit wenigen Modifikationen könnte diese Entscheidung auch dynamisch zur Laufzeit getroffen werden.
Erforderlich wäre die Implementierung von Marker"=Schnittstellen und die Annotation von \textit{Qualifier}, um dem \ac{CDI}"=Mechanismus die Auflösung der Abhängigkeiten zu ermöglichen. 
Dies vermischt allerdings Geschäftslogik mit Bereitstellungsdetails und führt zu einer engen Kopplung dieser Komponenten.
Mit \ac{CDI} 1.1 könnte durch das Überschreiben eines \textit{InjectionPoints} diese Einschränkung aufgehoben werden \cite{weld}.

% Literaturangaben
\bibliography{literatur/Literatur}
\bibliographystyle{unsrt}


\end{document}
