\section{Fazit und Ausblick}
\begin{table}[t]
\setlength{\tabcolsep}{10pt}
\centering
\caption{\label{tab:zielerreichung}Zielerreichung durch die Beispielimplementierung.}
\begin{tabular}{lc}
Methodensignaturen & \checkmark \\
Serialisierung     & \checkmark \\
Exceptions         & \checkmark / 0\\
\end{tabular}
\vspace{-10pt}
\end{table}
Das grundsätzliche Ziel des \ac{CDR}"=Entwurfsmusters ist es, die Geschäftslogik von den Details der Bereitstellung zu trennen.
Tabelle \ref{tab:zielerreichung} gibt einen Überblick darüber, für welche der in Kapitel \ref{sec:introduction} definierten Bereiche dieses Ziel erreicht werden konnten.
Die Bereiche \textbf{Methodensignaturen} und \textbf{Serialisierung} sind in der Beispielimplementierung frei von Bereitstellungsdetails. 
Das gesetzte Ziel konnte hier erreicht werden.\\
Einschränkungen gibt es bei der Fehlerbehandlung. Fehlertypen können zwar auch bei entfernter Bereitstellung erhalten bleiben, aber Kapitel \ref{sec:implementation} zeigt, dass für eine erfolgreiche Abbildung der Fehlertypen auf \ac{HTTP}"=Statuscodes und wieder zurück, die Fehlertypen den Typ \textit{RuntimeException} spezialisieren müssen.
Da \textit{RuntimeExceptions} in Java eine besondere Semantik haben, schränkt dies die Fehlerbehandlung der Anwendung ein und ist zu vermeiden. 
Den Ursprung hat diese Einschränkung in der genutzten \textit{resteasy}"=Version. Andere \ac{JAX-RS}"=Implementierungen haben diese Einschränkung nicht. Siehe dazu auch \cite{cxf}.\\
Die Beispielimplementierung legt zum Zeitpunkt der Bereitstellung der Anwendung fest, ob eine von CDR aktivierte Komponente lokal oder entfernt referenziert wird. 
Mit wenigen Modifikationen könnte diese Entscheidung auch dynamisch zur Laufzeit getroffen werden.
Erforderlich wäre die Implementierung von Marker"=Schnittstellen und die Annotation von \textit{Qualifier}, um dem \ac{CDI}"=Mechanismus die Auflösung der Abhängigkeiten zu ermöglichen. 
Dies vermischt allerdings Geschäftslogik mit Bereitstellungsdetails und führt zu einer engen Kopplung dieser Komponenten.
Mit \ac{CDI} 1.1 könnte durch das Überschreiben eines \textit{InjectionPoints} diese Einschränkung aufgehoben werden \cite{weld}.