\section{Umsetzung\label{sec:implementation}}
\begin{table}
\caption{\label{tab:project}Projektaufbau der Implementierung}
\begin{tabular}{lp{2cm}p{8cm}}
cdr-main &             & \\
|\_      & cdr-lib     & \ac{CDR}"=Komponenten.\\
|\_      & service-api & \textit{Service}"=Schnittstelle.\\
|\_      & service-lib & Implementierung der \textit{Service}"=Schnittstelle.\\
|\_      & cdr-poc     & Beinhaltet die Testfälle zur Veranschaulichung des \ac{CDR}"=Entwurfsmusters und die für \textit{Service} spezifische \textit{CDR hooks}.\\
\end{tabular}
\end{table}
Eine Beispielimplementierung des \ac{CDR}"=Entwurfsmusters in Java ist unter \href{https://github.com/Robotregent/cdr}{https://github.com/Robotregent/cdr} zu finden. 
Tabelle \ref{tab:project} zeigt den Aufbau des Mavenprojekts und den Zweck der einzelnen Module.
Die Funktionsweise der Beispielimplementierung wird an Hand zweier Testfälle aufgezeigt. Der erste Testfall zeigt wie eine lokale Bereitstellung für die Beispielkomponenten aufgebaut ist, der zweite Testfall zeigt die entfernte Bereitstellung nach dem \ac{CDR}"=Entwurfsmusters.
Um die Testfälle ausführen zu können, muss ein JBoss Anwendungsserver in der Version 7.1.1\footnote{\href{http://download.jboss.org/jbossas/7.1/jboss-as-7.1.1.Final/jboss-as-7.1.1.Final.zip}{{http://download.jboss.org/jbossas/7.1/jboss-as-7.1.1.Final/jboss-as-7.1.1.Final.zip}}} verfügbar sein.  
Der Befehl \colorbox{mygray}{\lstinline!mvn install!}, ausgeführt im Ordner \textit{cdr-main}, installiert alle restlichen Abhängigkeiten, übersetzt das Projekt und führt die Testfälle aus.\\
Im Folgenden werden die notwendigen Anpassungen für eine Umsetzung des \ac{CDR}"=Entwurfsmusters aufgezeigt.
\paragraph{Aktivierung}
Um eine Schnittstelle für das \ac{CDR}"=Entwurfsmuster zu aktivieren, muss diese mit \ac{JAX-RS} Annotationen erweitert werden. 
Beispielcode \ref{lst:activation} zeigt dies für die \textit{Service}"=Komponente.
\begin{lstlisting}[caption={Aktivierung},captionpos=b,label=lst:activation] 
@Path("/")
public interface Service {	
	@GET
	@Path("/item")
	@Produces(MediaType.APPLICATION_JSON)
	public TodoItem[] getItems();	

	@POST
	@Path("/item")
	@Consumes(MediaType.APPLICATION_JSON)
	public Integer takeItem(TodoItem tdi);	
}
\end{lstlisting}
Die Methoden der Komponente werden mit Hilfe der Annotationen \colorbox{mygray}{\lstinline!@GET!} und \colorbox{mygray}{\lstinline!@POST!} den entsprechenden \ac{HTTP}"=Methoden zugeordnet. \colorbox{mygray}{\lstinline!@Path!} bestimmt unter welcher \ac{URI} die annotierten Elemente zu erreichen sind. Weiterführende Informationen über die Gestaltung von \ac{REST}"=Schnittstellen und die Implementierung dieser mit \ac{JAX-RS} liefern Tilkov \cite{Tilkov2011} und Burke \cite{Burke2010}. \\
Die Annotationen \colorbox{mygray}{\lstinline!@Consumes!} und \colorbox{mygray}{\lstinline!@Produces!} steuern die Serialisierung der Parameter und werden im entsprechenden Abschnitt noch näher behandelt.
\newpage
\paragraph{Stellvertreterobjekt}
Der Beispielcode \ref{lst:factory} zeigt mit einem Ausschnitt der abstrakte Klasse \colorbox{mygray}{\lstinline!CDRFactory <T>!} die zentrale Komponente im \ac{CDR}"=Entwurfsmuster.
Die Methode \colorbox{mygray}{\lstinline!T produces()!} muss für jede Ausprägung der abstrakten Klasse implementiert werden und agiert als \textit{Producer}"=Methode des \ac{CDI}"=Mechanismus. Bei der Implementierung genügt es, die Methode \colorbox{mygray}{\lstinline!T getService (Class<T> clazz)!} aufzurufen, um das Stellvertreterobjekt zu erzeugen. 
Die \ac{CDI}"=Spezifikation legt in §3.3 fest, dass eine \textit{Producer}"=Methode keine Typvariable als Rückgabewert haben darf. Daher ist diese Indirektion notwendig. Siehe dazu auch \cite{group2009jsr299}.\\
In der Beispielimplementierung wird die \ac{URI} der entfernten Komponente über eine \textit{Properties}"=Datei aufgelöst. Hier sind auch andere Mechanismen wie z.B. Annotationen denkbar.\\ 
Die Methoden \colorbox{mygray}{\lstinline!registerClientInterceptor()!} und \colorbox{mygray}{\lstinline!registerClientExceptionMapper()!} sind die \textit{CDR hooks} Für den Client. 
\begin{lstlisting}[caption={Abstrakte CDR"=Fabrik},captionpos=b,label=lst:factory] 
public abstract class CDRFactory <T> {	

	protected abstract T produces();

	protected T getService (Class<T> clazz){							
				
		URI uri = getUri(clazz.getCanonicalName());			
		ClientRequestFactory crf = new ClientRequestFactory(UriBuilder.fromUri(uri).build());	
		
		registerClientInterceptor(crf);	
		
		ResteasyProviderFactory pf = ResteasyProviderFactory.getInstance();
		
		registerClientExceptionMapper(pf);		

		return crf.createProxy(clazz);
	}
	
	protected void registerClientInterceptor(ClientRequestFactory crf){
	}
	
	protected void registerClientExceptionMapper(ResteasyProviderFactory pf){
	}
}
\end{lstlisting}
\newpage
\paragraph{Methodensignaturen}
Die Konfiguration der \ac{HTTP}"=Kommunikation erfolgt bei \ac{JAX-RS} durch Annotation der Komponenten. Dies ermöglicht es, die Geschäftslogik frei von den Details der Bereitstellung zu halten.
Allerdings muss hier eine Abwägung zwischen Transparenz der Methodensignaturen und Aussagekraft der \ac{REST}"=Schnittstelle getroffen werden.
Die Beispielimplementierung nutzt die Möglichkeiten von \ac{JAX-RS} und \textit{resteasy}, um \ac{HTTP}"=Anfragen und "=Antworten unabhängig von der Geschäftslogik manipulieren zu können. Im Konzept werden diese Komponenten als \textit{CDR hooks} bezeichnet. 
Durch diesen Mechanismus können beide Ziele erreicht werden.\\
Das Manipulieren der \ac{HTTP}"=Anfrage ist über die in Beispielcode \ref{lst:factory} gezeigte Methode \colorbox{mygray}{\lstinline!registerClientInterceptor()!} möglich. Hier kann ein \textit{resteasy ClientExecutionInterceptor} registriert werden. Siehe dazu auch \cite{resteasy}.\\
Beispielcode \ref{lst:serverinterceptor} zeigt, wie die \ac{HTTP}"=Antwort durch einen \textit{CDR hook} auf der Serverseite manipuliert wird. Um die Aussagekraft der \ac{REST}"=Schnittstelle zu vergrößern, wird nach dem Erzeugen einer Ressource der entsprechende \ac{HTTP}"=Statuscode und \textit{Location"=Header} gesetzt. 
\begin{lstlisting}[caption={ServerPostProcessInterceptor},captionpos=b,label=lst:serverinterceptor] 
@Provider
@ServerInterceptor
public class ServerPostProcessInterceptor implements PostProcessInterceptor, AcceptedByMethod {	
	
	@Context
	HttpServletRequest request;
	
	public void postProcess(ServerResponse response) {
		Integer id = (Integer) response.getEntity();
		response.setStatus(201);
		MultivaluedMap<String, Object> headers = response.getMetadata();
		 
		String url = request.getRequestURL().toString();
		
		headers.add("Location", url );
	}
}
\end{lstlisting}

\paragraph{Exceptions}
Damit die Fehlerbehandlung transparent zur Bereitstellung bleibt, müssen die entsprechenden \textit{CDR hooks} implementiert werden. Ohne diese zusätzlichen Komponenten erzeugt die \ac{JAX-RS}"=Laufzeitumgebung nur eine generische Fehlernachricht. 
Der \ac{HTTP}"=Statuscode wäre 500 und auf der Seite des Clients würde der \textit{resteasy}"=Fehlertyp \textit{ClientResponseFailure} erzeugt werden. 
Dadurch würde die Semantik der Fehlerbehandlung in der Anwendung verloren gehen und die Geschäftslogik wäre nicht mehr frei von den Details der Bereitstellung. Diese Problematik kann durch die folgenden \textit{CDR hooks} umgangen werden:
\begin{description}
\item [ServerExceptionMapper] Wird von der \ac{JAX-RS}"=Laufzeitumgebung aufgerufen, wenn der entsprechende Fehler auftritt. Der Fehlertyp kann durch einen individuellen Statuscode in der \ac{HTTP}"=Antwort repräsentiert werden.
\item [ClientExceptionMapper] In der \ac{JVM} des Clients kann der Statuscode ausgelesen und der entsprechende Fehlertyp erzeugt werden. Allerdings ist dies nur für Fehlertypen möglich, die den Typ \textit{RuntimeExpetion} erweitern. Beispielcode \ref{lst:ClientExceptionMapper} zeigt einen Ausschnitt aus der Implementierung.
\end{description}
\begin{lstlisting}[caption={ClientExceptionMapper},captionpos=b,label=lst:ClientExceptionMapper] 
public class ClientExceptionMapper implements ClientErrorInterceptor{

	public void handle(ClientResponse response) throws RuntimeException{
			
		if (response.getStatus() == 523){
			throw new SomeApplicationException();
		}
	}
}
\end{lstlisting}
\paragraph{Serialisierung}
Die Annotationen \colorbox{mygray}{\lstinline!@Consumes!} und \colorbox{mygray}{\lstinline!@Produces!} steuern die Serialisierung der Parameter und Rückgabewerte durch die \ac{JAX-RS}"=Laufzeitumgebung. 
Werden für diese Annotationen die Werte \textit{application/xml} oder \textit{application/json} gesetzt, kann die \ac{JAX-RS}"=Laufzeitumgebung alle Klassen serialisieren, die entsprechend dem \ac{JAXB}"=Standard annotiert sind.  
\newpage