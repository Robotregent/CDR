\section{Einleitung\label{sec:introduction}}
In verteilten Softwarearchitekturen ist es wichtig, dass die einzelnen Komponenten flexibel auf verschiedenen Server bereitgestellt werden können. 
Je nach Anwendungsfall müssen beispielsweise alle Komponenten auf einem Server oder eine einzelne Komponente auf einem speziellen Server bereitgestellt werden. 
Die Nutzung einer Komponenten innerhalb der Geschäftslogik der Anwendung soll dabei frei von den Details der Verteilung sein.\\
In einer objektorientierten Sprache wie Java müssen folgende drei Aspekte berücksichtigt werden:
\begin{description}
\item[Methodensignaturen] Die Parameter und der Rückgabewert der Methoden sollen keine Details über die Art der Bereitstellung aufzeigen.
\item[Serialisierung] Für eine flexibel Verteilung der Komponenten müssen die Objekte, die als Parameter und Rückgabewert genutzt werden, serialisierbar sein. Die Details der Serialisierung sollen nicht Teil der Geschäftslogik sein. 
\item[Exceptions] Auch die Behandlung von Ausnahmefehlern muss frei von Bereitstellungsdetails bleiben. Zudem muss sichergestellt werden, dass auch bei verteilten Bereitstellungsvarianten die speziellen Fehlerklassen der Anwendung erhalten bleiben.    
\end{description}
Das im Folgenden vorgestellte Konzept \ac{CDR} versucht diese Anforderungen für die \ac{Java EE}"=Plattform umzusetzen.